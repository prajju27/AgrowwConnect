% Options for packages loaded elsewhere
\PassOptionsToPackage{unicode}{hyperref}
\PassOptionsToPackage{hyphens}{url}
%
\documentclass[
]{article}
\usepackage{amsmath,amssymb}
\usepackage{lmodern}
\usepackage{iftex}
\ifPDFTeX
  \usepackage[T1]{fontenc}
  \usepackage[utf8]{inputenc}
  \usepackage{textcomp} % provide euro and other symbols
\else % if luatex or xetex
  \usepackage{unicode-math}
  \defaultfontfeatures{Scale=MatchLowercase}
  \defaultfontfeatures[\rmfamily]{Ligatures=TeX,Scale=1}
\fi
% Use upquote if available, for straight quotes in verbatim environments
\IfFileExists{upquote.sty}{\usepackage{upquote}}{}
\IfFileExists{microtype.sty}{% use microtype if available
  \usepackage[]{microtype}
  \UseMicrotypeSet[protrusion]{basicmath} % disable protrusion for tt fonts
}{}
\makeatletter
\@ifundefined{KOMAClassName}{% if non-KOMA class
  \IfFileExists{parskip.sty}{%
    \usepackage{parskip}
  }{% else
    \setlength{\parindent}{0pt}
    \setlength{\parskip}{6pt plus 2pt minus 1pt}}
}{% if KOMA class
  \KOMAoptions{parskip=half}}
\makeatother
\usepackage{xcolor}
\IfFileExists{xurl.sty}{\usepackage{xurl}}{} % add URL line breaks if available
\IfFileExists{bookmark.sty}{\usepackage{bookmark}}{\usepackage{hyperref}}
\hypersetup{
  hidelinks,
  pdfcreator={LaTeX via pandoc}}
\urlstyle{same} % disable monospaced font for URLs
\setlength{\emergencystretch}{3em} % prevent overfull lines
\providecommand{\tightlist}{%
  \setlength{\itemsep}{0pt}\setlength{\parskip}{0pt}}
\setcounter{secnumdepth}{-\maxdimen} % remove section numbering
\ifLuaTeX
  \usepackage{selnolig}  % disable illegal ligatures
\fi

\author{}
\date{}

\begin{document}

\begin{quote}
\textbf{Introduction}
\end{quote}

\textbf{1.1 General}

Agriculture is the practice of cultivating plants and livestock in order
to provide facilities the human beings. In the rise of the sedentary
human lifestyle agriculture was the key development. The cultivation of
plant and food grains began years ago in order to provide food to the
city population. farmers play a crucial role in ensuring food security.
However, a significant gap often exists between farmers and end
consumers, leading to inefficiencies and missed opportunities.

Agriculture is an essential sector that sustain s human life and
economic development. With the advent of technology, there is a pressing
need to integrate modern solutions to address challenges faced by
farmers and enhance agricultural productivity. agriculture sector also
known as primary sector is essential for economic growth in any economy
including India. It has emerged as the essential-growing sector in the
global economy since independence. It was seen that the highest
percentage of household and land holdings were found in marginal
category with 75.41 percent and 29.75 per cent followed by small
category.

Only 0.24 per cent of household has a land holding of 5.81 per cent
under large category. India is the largest agricultural powerhouse
worldwide and the leading producer of spices, pulses, and milk. Not only
that, our country has the largest area that is used to cultivate coon,
wheat, and rice. Agricultural held almost 75 per cent share in India's
GDP a few decades ago World Bank, 2020. To this day, the share has gone
down to around 18 per cent.

\textbf{1.2 Literature survey}

Graf, A. and Maas, P. {[}1{]} The value concept is one of marketing
theory's basic elements. Identifying and creating customer value (CV) -
understood as value for customers - is regarded as an essential
prerequisite for future company success. Nevertheless, not until quite
recently has CV received much research attention. Ideas on how to
conceptualize and link the concept to other constructs vary widely. The
literature contains a multitude of different definitions, models, and
measurement approaches. This article provides a broad overview,
analysis, and critical evaluation of the different trends and approaches
found to date in this research field, encompassing the development of
perceived and desired customer value research, the relationships between
the CV construct and other central marketing constructs, and the linkage
between CV and the company interpretation of the value of the customer,
like customer lifetime value (CLV). The article concludes by pointing
out some of the challenges this field of research will face in the
future.

Halinen, A. and To¨ rnroos, J. {[}2{]} Business conditions have changed
significantly since the first notions of networks in industrial
marketing were made in the early 1980s. Globalization and the so-called
``new economy'' are based on continuous change, the increasing use of
information and communication technologies and complex networks of
relationships between firms. This being the case, we should look more
closely at how networks are constituted and how they currently function.
This paper gives perspectives on how to conduct case research for
understanding contemporary business networks. Four imminent challenges
of case research that aims at theory development are discussed: the
problem of network boundaries, network complexity, the role of time and
case comparisons. Potential tools and ideas for solving these problems
are put forward.

Ramirez, R. {[}3{]} This paper surveys the history of an alternative
view of value creation to that associated with industrial production. It
argues that technical breakthroughs and social innovations in actual
value creation render the alternative-a value co-production
framework-ever more pertinent. The paper examines some of the
implications of adopting this framework to describe and understand
business opportunity, management, and organizational practices. In the
process, it reviews the research opportunities a value co-production
framework opens up.

Ulaga, W. and Chacour, S. {[}4{]} Delivering superior value to customers
is an ongoing concern of management in many business markets of today.
Knowing where value resides from the standpoint of the customer has
become critical for suppliers. In this article, the construct of
customer-perceived value is first assessed through a literature review.
Then a multiple-item measure of customer value is developed, and our
approach is illustrated by the marketing strategy development project of
a major chemical manufacturer in international markets. The article
finally discusses how the customer value audit can be linked to
marketing strategy development and provides guidelines for managerial
actions.

\textbf{2.Problem statement}

\begin{enumerate}
\def\labelenumi{\arabic{enumi}.}
\item
  \textbf{Market Research}: Conduct thorough research to understand the
  needs and pain points of both farmers and customers in the target
  market. Identify challenges such as limited access to markets, lack of
  information about produce, price fluctuations, etc.
\item
  \textbf{Stakeholder Analysis:} Identify key stakeholders including
  farmers, customers, wholesalers, retailers, and any intermediaries
  involved in the agricultural supply chain.
\end{enumerate}

The agricultural sector faces a huge problem in bridging the gap between
producers and customers, preventing efficient distribution and access to
fresh products. This disparity is caused by a variety of causes,
including limited direct communication channels, a lack of transparency
in pricing and product availability, and inefficient distribution
networks.

Farmers frequently struggle to reach clients directly, relying on
intermediaries who may abuse market imbalances, resulting in unjust
pricing and lower profitability for both sides. Furthermore, customers
have difficulty acquiring credible information about the origin,
quality, and pricing of agricultural products, which leads to poor
purchase decisions and decreased trust in the supply chain.

Tap has an influence not only on farmers' economic sustainability, but
also on customers' access to healthful, locally sourced food. To address
this issue, innovative solutions must be developed that use technology
to facilitate direct communication between farmers and customers,
increase transparency in pricing and product information, and streamline
distribution channels to ensure equitable and efficient market access
for all stakeholders.

\textbf{2.1 Objective}

The project aims to bridge the gap between farmers and customers by
establishing a direct connection, fostering mutual understanding, and
enhancing the overall agricultural ecosystem.

\begin{enumerate}
\def\labelenumi{\arabic{enumi}.}
\item
  \textbf{Market Accessibility:} Facilitate easy access to markets for
  farmers by connecting them directly with customers, eliminating
  intermediaries, and ensuring fair pricing mechanisms.
\item
  \textbf{Information Exchange}: Create a platform for the exchange of
  information between farmers and customers regarding produce
  availability, quality, pricing, and farming practices, thereby
  fostering transparency and trust.
\item
  \textbf{Customer Engagement:} Engage customers in the agricultural
  process by offering insights into farming practices, the journey of
  produce from farm to table, and opportunities for direct interaction
  with farmers.
\item
  \textbf{Community Building:} Foster a sense of community among farmers
  and customers, encouraging collaboration, sharing of experiences, and
  support networks.
\item
  \textbf{Facilitating Communication and Collaboration:} Effective
  communication and collaboration are essential for fostering mutually
  beneficial relationships between farmers and customers.
\end{enumerate}

This objective involves establishing channels for direct communication
between farmers and consumers, such as online forums, social media
platforms, and community events.

By facilitating dialogue and feedback, both parties can better
understand each other's needs and preferences, leading to improved
product offerings and customer satisfaction.

\textbf{3. Methodology}

This methodology offers a structured approach to addressing the
challenges of connecting farmers directly with customers, combining
technical innovation.

\textbf{Framer specification :}

The primary goal of this project is to provide a web application that
allows farmers to manage their specifications. The application will
serve as a central repository for storing and accessing farmer
information, such as their name, farmer ID, location, mobile number, and
password. The system will also include secure login and authorization
systems to safeguard farmer information.

\begin{itemize}
\item
  \textbf{Functional Requirements}
\end{itemize}

\textbf{User Registration}: Farmers can register by providing their
name, location, mobile number, and password.

\textbf{User Login:} Registered farmers can log in using their mobile
number and password.

\textbf{Search Functionality}: Users can search for farmers by name,
location, or farmer ID.

\textbf{Profile Management}: Farmers can update their personal details
and password.

\begin{itemize}
\item
  \textbf{Non Functional Requirements}
\end{itemize}

\textbf{Security:} Put strong security measures in place to keep user
information safe.

\textbf{Usability:} Make sure the program is easy to use and readily
available.

\textbf{Performance:} The system must be quick to respond and able to
support multiple users at once.

\textbf{Scalability:} In order to handle increasing amounts of data and
users, the program must be scalable.

\textbf{Customer Specification :}

Creating a web project to manage client specifications, including
capturing facts like name, address, mobile number, and password,
necessitates an organized method to ensure all criteria are completed
promptly and securely. The following is a complete process for carrying
out this online project:

\textbf{Functional requirements :}

\textbf{User Registration:} Users can create an account by providing
their name, location, mobile number, The system should validate the
uniqueness of the location.

\textbf{User Login:} Users can log in using their mobile number and
location.

\textbf{Profile Management}: Users can view and update their profile
information.

\textbf{Non Functional requirements :}

\textbf{Performance:} The application should load within 2 seconds under
normal load conditions. The system should handle at least 100 concurrent
users.

\textbf{Scalability:} The application should be designed to scale
horizontally to accommodate increased load.

\textbf{Usability:} The user interface should be intuitive and
responsive across devices.

\textbf{Reliability:} The system should have 99.9\% uptime.

\textbf{4. Implementation}

\textbf{Implementation}

\begin{itemize}
\item
  \textbf{Frontend Development:}
\end{itemize}

\begin{quote}
Develop registration, login, profile management, and admin dashboard
pages.
\end{quote}

\begin{itemize}
\item
  \textbf{Backend Development:}
\end{itemize}

\begin{quote}
Implement RESTful APIs for user registration, login, profile management,
and admin functionalities.
\end{quote}

\begin{itemize}
\item
  \textbf{Database Integration:}
\end{itemize}

\begin{quote}
Implement data access layer to interact with the database.
\end{quote}

\textbf{References}

{[}1{]} Graf, A. and Maas, P. (2008), ``Customer value from a customer
perspective: a comprehensive review'', Journal fu¨r Betriebswirtschaft,
Vol. 58 No. 1, pp. 1-20.

{[}2{]} Halinen, A. and To¨ rnroos, J. (2005), ``Using case methods in
the study of contemporary business networks'', Journal of Business
Research, Vol. 58 No. 9, pp. 1285-97.

{[}3{]} Ramirez, R. (1999), ``Value co-production: intellectual origins
and implications for practice and research'', Strategic Management
Journal, Vol. 20 No. 1, pp. 49-65.

{[}4{]} Ulaga, W. and Chacour, S. (2001), ``Measuring customerperceived
value in business markets'', Industrial Marketing Management, Vol. 30
No. 6, pp. 525-40.

{[}5{]} Strauss, A.C. and Corbin, J.M. (1990), Basics of Qualitative
Research: Grounded Theory Procedures and Techniques, Sage Publications,
London.

{[}6{]} Lindgreen, A. and Wynstra, F. (2005), ``Value in business
markets: what do we know? Where are we going?'', Industrial Marketing
Management, Vol. 34 No. 7, pp. 732-48.

{[}7{]} Patton, M.Q. (1990), Qualitative Evaluation and Research
Methods, Sage Publications, Newbury Park, CA.

{[}8{]} Jalkala, A., Cova, B., Salle, R. and Salminen, R.T. (2010),
``Changing project business orientations: towards a new logic of project
marketing'', European Management Journal, Vol. 28 No. 2, pp. 124-38.

{[}9{]} Ramirez, R. {[}5{]} Woodruff, R.B. (1997), ``Customer value: the
next source for competitive advantage'', Academy of Marketing Science,
Vol. 25 No. 2, pp. 139-53.

\end{document}
